% ==========================================
% BAB I PENDAHULUAN
% ==========================================
\chapter{PENDAHULUAN}
\label{chap:pendahuluan}
% --- Latar Belakang ---
\section{Latar Belakang}
Hipertensi merupakan tantangan kesehatan masyarakat global dan faktor risiko utama yang berkontribusi pada morbiditas dan mortalitas akibat penyakit kardiovaskular, seperti stroke dan gagal jantung (Reges et al., 2020). Data WHO (2023) mengindikasikan bahwa 1,28 miliar orang dewasa di seluruh dunia hidup dengan hipertensi, sebuah angka yang sejalan dengan tingginya prevalensi di Indonesia. Permasalahan fundamental dalam manajemen hipertensi saat ini terletak pada paradigmanya yang bersifat reaktif. Intervensi klinis sering kali baru dilakukan setelah diagnosis ditegakkan atau setelah terjadi fluktuasi tekanan darah (TD) yang signifikan (Savoia et al., 2017; Carey et al., 2018).. Kelemahan dari pendekatan reaktif ini adalah kegagalannya dalam mengantisipasi dan mencegah episode akut yang paling berbahaya yaitu krisis hipertensi. Kondisi darurat medis ini terjadi ketika tekanan darah sistolik dan diastolik melebihi 180/120 mmHg, yang secara langsung berpotensi menyebabkan kerusakan pada organ target (Talle et al., 2023)
\begin{enumerate}
\item	Kondisi atau situasi topik yang dibahas beserta permasalahannya, misalnya tentang pengelolaan informasi di puskesmas daerah pedesaan dan masalah yang dihadapi.
\item	Berbagai solusi yang telah diterapkan atau solusi yang tersedia dan memungkinkan untuk diterapkan untuk menyelesaikan masalah tersebut.
\end{enumerate}

% --- Rumusan Masalah ---
\section{Rumusan Masalah}
Rumusan Masalah berisi masalah utama yang dibahas dalam tugas akhir. Rumusan masalah yang baik memiliki struktur sebagai berikut:
\begin{enumerate}
\item	Pokok persoalan dari kondisi atau situasi yang ada saat ini. Dengan kata lain, jelaskan kelemahan atau kekurangan dari kondisi, situasi, atau solusi yang dijelaskan pada latar belakang. Ini merupakan pokok rumusan masalah.
\item	Elaborasi lebih lanjut urgensi penyelesaian masalah tersebut (mengapa penting untuk diselesaikan dan akibat yang dapat terjadi jika tidak diselesaikan).
\item	Usulan singkat terkait dengan solusi yang ditawarkan untuk menyelesaikan persoalan.
Penting untuk diperhatikan bahwa persoalan yang dideskripsikan pada subbab ini akan dipertanggungjawabkan di bab Evaluasi (apakah terselesaikan atau tidak).
\end{enumerate}

% --- Tujuan ---
\section{Tujuan}
Tuliskan tujuan utama dan/atau tujuan detail yang akan dicapai dalam pelaksanaan tugas akhir. Fokuskan pada hasil akhir yang ingin diperoleh setelah tugas akhir diselesaikan, terkait dengan penyelesaian persoalan pada rumusan masalah. Penting untuk diperhatikan bahwa tujuan yang dideskripsikan pada subbab ini akan dipertanggungjawabkan di akhir pelaksanaan tugas akhir apakah tercapai atau tidak. Tuliskan kriteria keberhasilan tugas akhir ini.

% --- Batasan Masalah ---
\section{Batasan Masalah}
Tuliskan batasan-batasan yang diambil dalam pelaksanaan tugas akhir. Batasan ini dapat dihindari (bersifat opsional, tidak perlu ada) jika topik atau judul tugas akhir dibuat cukup spesifik.

% --- Metodologi Pengerjaan TA ---
\section{Metodologi}
Tuliskan semua tahapan yang akan dilalui selama pelaksanaan tugas akhir. Tahapan ini spesifik untuk menyelesaikan persoalan tugas akhir. Khusus untuk penyusunan proposal ini, jelaskan secara detail:
\begin{enumerate}
\item	Tahapan investigasi pengumpulan fakta di latar belakang untuk merumuskan masalah.
\item	Langkah-langkah pencarian, pengelompokan, dan penapisan literatur atau sumber informasi untuk mengumpulkan informasi yang relevan tentang topik yang diangkat, termasuk teori (konsep atau teori apa saja yang perlu dicari), hal-hal yang telah dicapai oleh orang lain (cara mencari dan kata kuncinya), dan berbagai informasi pendukung, untuk mencari solusi terhadap masalah yang dibahas. Gunakan metodologi yang tepat dalam menggali informasi dan dokumentasikan prosesnya (termasuk rekaman wawancara atau survei) di dalam Lampiran, termasuk tautan ke video atau foto. Hasil penggalian informasi ini akan dijelaskan secara sistematis di Bab II Studi Literatur.
\end{enumerate}